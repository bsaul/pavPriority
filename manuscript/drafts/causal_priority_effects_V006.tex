\documentclass[]{article}
\usepackage{lmodern}
\usepackage{amssymb,amsmath}
\usepackage{ifxetex,ifluatex}
\usepackage{fixltx2e} % provides \textsubscript
\ifnum 0\ifxetex 1\fi\ifluatex 1\fi=0 % if pdftex
  \usepackage[T1]{fontenc}
  \usepackage[utf8]{inputenc}
\else % if luatex or xelatex
  \ifxetex
    \usepackage{mathspec}
  \else
    \usepackage{fontspec}
  \fi
  \defaultfontfeatures{Ligatures=TeX,Scale=MatchLowercase}
\fi
% use upquote if available, for straight quotes in verbatim environments
\IfFileExists{upquote.sty}{\usepackage{upquote}}{}
% use microtype if available
\IfFileExists{microtype.sty}{%
\usepackage{microtype}
\UseMicrotypeSet[protrusion]{basicmath} % disable protrusion for tt fonts
}{}
\usepackage[margin=1in]{geometry}
\usepackage{hyperref}
\hypersetup{unicode=true,
            pdftitle={Causal Priority Effects},
            pdfborder={0 0 0},
            breaklinks=true}
\urlstyle{same}  % don't use monospace font for urls
\usepackage{graphicx,grffile}
\makeatletter
\def\maxwidth{\ifdim\Gin@nat@width>\linewidth\linewidth\else\Gin@nat@width\fi}
\def\maxheight{\ifdim\Gin@nat@height>\textheight\textheight\else\Gin@nat@height\fi}
\makeatother
% Scale images if necessary, so that they will not overflow the page
% margins by default, and it is still possible to overwrite the defaults
% using explicit options in \includegraphics[width, height, ...]{}
\setkeys{Gin}{width=\maxwidth,height=\maxheight,keepaspectratio}
\IfFileExists{parskip.sty}{%
\usepackage{parskip}
}{% else
\setlength{\parindent}{0pt}
\setlength{\parskip}{6pt plus 2pt minus 1pt}
}
\setlength{\emergencystretch}{3em}  % prevent overfull lines
\providecommand{\tightlist}{%
  \setlength{\itemsep}{0pt}\setlength{\parskip}{0pt}}
\setcounter{secnumdepth}{0}
% Redefines (sub)paragraphs to behave more like sections
\ifx\paragraph\undefined\else
\let\oldparagraph\paragraph
\renewcommand{\paragraph}[1]{\oldparagraph{#1}\mbox{}}
\fi
\ifx\subparagraph\undefined\else
\let\oldsubparagraph\subparagraph
\renewcommand{\subparagraph}[1]{\oldsubparagraph{#1}\mbox{}}
\fi

%%% Use protect on footnotes to avoid problems with footnotes in titles
\let\rmarkdownfootnote\footnote%
\def\footnote{\protect\rmarkdownfootnote}

%%% Change title format to be more compact
\usepackage{titling}

% Create subtitle command for use in maketitle
\newcommand{\subtitle}[1]{
  \posttitle{
    \begin{center}\large#1\end{center}
    }
}

\setlength{\droptitle}{-2em}

  \title{Causal Priority Effects}
    \pretitle{\vspace{\droptitle}\centering\huge}
  \posttitle{\par}
    \author{}
    \preauthor{}\postauthor{}
    \date{}
    \predate{}\postdate{}
  
\usepackage{mathtools}

\begin{document}
\maketitle
\begin{abstract}
TBD
\end{abstract}

\section{Introduction}\label{introduction}

The early bird gets the worm, but the second mouse gets the cheese.
Despite being poorly understood and studied, priority effects -- the
order and timing of species arrival -- have ecological consequences
(Fukami 2015). Priority effects invite counterfactual reasoning: if
species A arrives first would outcomes for species B be different
compared to a world where species B arrived first? Such counterfactual
questions are well suited for the potential framework of causal
inference (VanderWeele 2015).

To study within-host priority effects, Halliday, Umbanhowar, and
Mitchell (2017) conducted an experiment wherein three cohorts of
sentinel, pathogen-na\{:i\}ve tall fescue (\emph{Festuca arundinacea})
plants where placed in a field at three different timepoints during the
year. The timepoints were chosen to correspond to periods of high
prevelance during the annual epidemic cycles for each of three tall
fescue fungal pathogens: \emph{Colletotrichum} (early summer),
\emph{Rhizoctonia} (mid summer), and \emph{Puccinia} (late summer)
(Halliday, Umbanhowar, and Mitchell 2017).

A goal of this paper is to introduce experimental ecologist to the
causal inference methods of XXX, YYY, ZZZ. To do so, we begin by
expressing priority effects as functions of potential outcomes, thus
giving contrasts of potential outcomes a causal interpretation. We
discuss the assumptions necessary for identifying and estimating such
effects from observed data. Causal inference methods more familiar to
many ecologists such as structural equation models make certain implicit
assumptions, which, if violated, may lead to incorrect conclusions
(Imai, Keele, and Tingley 2010). After establishing the targets of
inference (estimands) and assumptions necessary for causal inference,
statistical models are parameterized to target the estimands and
estimatation and inference carried out.

\subsection{Priority Effects}\label{priority-effects}

Describe priority effects and importance to ecology (based on Fukami
(2015))

\begin{itemize}
\tightlist
\item
  Is hypothesis in this experiment niche preemption or niche
  modification? I think modification, though this complicated if a
  species can change from (e.g.) biotroph to necrotroph.
\item
  link priority effects to causal mediation:
\item
  mediation - how an effect occurs; interaction - for whom an effect
  occurs (VanderWeele 2015)
\item
  would be nice to map one of the figures in Fukami (2015) to the causal
  mediation SWIT
\end{itemize}

Wrinkle with this study/data

\begin{itemize}
\tightlist
\item
  randomized exposure (time of field placement) is essentially an
  instrumental variable
\item
  interval censored data
\item
  is death of leaf a censoring event or competing risk? There's a
  complication that leaves only live a certain period of time - during
  which they are either infected or not - but it is not as if a leaf
  \emph{could} survive for longer during which time we could observe the
  outcome. Then there's the fact that leaves in the last cohort can't be
  observed for the same length of time as other cohorts due to end of
  study and senesence.
\end{itemize}

\subsection{Causal inference using potential
outcomes}\label{causal-inference-using-potential-outcomes}

Structural equation models (Shipley 2016) are popular in biology and
ecology; however, parameters from these models may not carry a causal
interpretation without assumptions often implicitly made (Imai, Keele,
and Tingley 2010).

\begin{itemize}
\tightlist
\item
  the approach we describe starts from causal questions and works
  towards statistical ``answers'' (i.e.~parameter estimates, confidence
  intervals, hypothesis tests, etc)
\item
  the typical approach, however, is to take statistical ``answers'' and
  assume they are causal
\item
  a more principled approach is warranted and available
\item
  Imai, Keele, and Tingley (2010) argue the structural equation modeling
  approach to mediation ``is problematic for 3 reasons: the lack of a
  general definition of causal mediation effects independent of a
  particular statistical model, the inability to specify the key
  identification assumption, and the difficulty of extending the
  framework to nonlinear models.''
\end{itemize}

Problems with linear structural equation models (Imai, Keele, and
Tingley 2010):

\begin{itemize}
\tightlist
\item
  within LSEM, causal mediation effects are defined relative to
  particular statistical models
\item
  not generalization to nonlinear models {[}is this true for the SEM
  framework?{]}
\end{itemize}

An advantage of the potential outcome approach is that the causal
effects can be defined, identified, and estimators derived without
reference to specific statistical models (Imai, Keele, and Tingley
2010).

\subsection{Organization}\label{organization}

The paper is organized as follows:

\begin{itemize}
\tightlist
\item
  causal model for priority effects

  \begin{itemize}
  \tightlist
  \item
    experimental description
  \item
    notation
  \item
    graphic model
  \item
    defining causal estimands and clarifying the link between these and
    priority effects
  \item
    identifiability
  \item
    limitations and ``gotchas'' of this section
  \end{itemize}
\item
  statistical model

  \begin{itemize}
  \tightlist
  \item
    why this model?
  \item
    estimator
  \item
    limitation and ``gotchas'' with this model/estimator
  \end{itemize}
\item
  results

  \begin{itemize}
  \tightlist
  \item
    numerical results and interpretation
  \item
    stability and sensitivity checks
  \end{itemize}
\item
  discussion

  \begin{itemize}
  \tightlist
  \item
    what do these results mean?
  \item
    a call for more causal in causal
  \item
    suggestions for better designs and how identifiability results might
    suggest better design
  \item
    links to other causal frameworks (such as sufficient causes)
  \end{itemize}
\end{itemize}

\section{A Causal Model for Priority
Effects}\label{a-causal-model-for-priority-effects}

\begin{itemize}
\tightlist
\item
  First describe the target causal estimand; introduce notation
\item
  Then describe an idealized randomized experiment from which the
  estimand could be straightfowardly estimated
\item
  Then describe issues with this experiment (why it is difficult to
  carry out)
\item
  Then describe the experiment that was conducted in Halliday,
  Umbanhowar, and Mitchell (2017)
\end{itemize}

\subsection{Experiment}\label{experiment}

Describe the experiment that was run:

\begin{itemize}
\tightlist
\item
  experimental treatments
\item
  what was measured
\end{itemize}

Describe study predictions or hypotheses.

\subsection{Notation}\label{notation}

\begin{itemize}
\tightlist
\item
  Define units
\item
  Define variables
\end{itemize}

Let \(A = \{0, 1\}\) be exposure to either low prevelance (0) or high
prevelance (1) of \emph{Collectotricum}. Let \(M(a)\) be the binary
indicator of visible \emph{Collectotricum} symptoms under exposure
setting \(a\) during the first four weeks after leaf emergence. Let
\(Y(a, m)\) be the binary indicator of visible \emph{Rhizoctonia}
symptoms four weeks after leaf emergence.

\begin{itemize}
\tightlist
\item
  Is \(M(a)\) time to infection or just an indicator of infection?
  Latter is probably easier to deal with.
\item
  Is \(M(a)\) infection by either collectotricum or rhizoctonium (or
  nothing) (i.e.~a competing risks problem) or just infection by, say,
  just collectotricum or not?
\end{itemize}

Assumes no-interference. Plausible here? Are units far enough away in
space such that treatment of one unit does not effect outcome in another
unit?

\subsection{Graphical Model}\label{graphical-model}

The more algebraic potential outcomes (sometimes referred to as the
counterfactual or Rubin causal model {[}holland1986{]}) and directed
acyclic graphs (DAGs) have sometimes been viewed as different approaches
for understanding causality. We see both approaches as useful and we use
single world intervention graphs (SWIGs) or single world intervention
templates (SWITs) (Richardson and Robins 2013) to explicitly represent
potential outcomes on a graph

\subsection{Possible Estimands - targets of
inference}\label{possible-estimands---targets-of-inference}

First we layout a generic framework defining causal effects before
specifying priority causal effects.

A causal effect is in general a contrast between a functional, denoted
\(\mathcal{F}\), of the distribution of potential outcomes under
different settings.

\[
CE[Y(a, m), Y(a’, m’); \mathcal{F}]
\]

Often \(\mathcal{F} \coloneqq \mathrm{E}(\cdot)\), i.e., an expectation,
but other features of the distribution such as quantiles or survival
curves may be of interest too. The contrast function may be the
difference, ratio, odds ratio, or whatever.

Here, we follow Imai, Keele, and Tingley (2010) in defining causal
mediation effects.

Average direct effect of treatment:

\[
DE(a, a’, a^{\star}) = \mathrm{E}[Y(a, M(a^{\star})) - Y(a’, M(a^{\star}))]
\]

This the effect of setting treatment from \(a\) to \(a’\) while the
mediator takes the value it would take under \(a^{\star}\), allowing for
\(a^{\star} = a\) or \(a’\).

Average causal mediation effect (the effect of the treatment through the
mediator):

\[
ME(a, a’, a^{\star}) = \mathrm{E}[Y(a, M(a’)) - Y(a, M(a^{\star}))]
\]

For example, \(ME(0, 1, 1) = \mathrm{E}[Y(0, M(1)) - Y(0, M(0))]\) asks
the question: what would be the effect on the outcome of changing the
mediator from its value under control to its value under SA while
holding the treatment at the control value?

A table or figure describing the link between hypotheses and causal
effects:

\begin{itemize}
\tightlist
\item
  Fletcher: lay out hypotheses about effects of SA and JA and how you
  expect these effects to be mediated.
\end{itemize}

\subsection{Identifiability
Assumptions}\label{identifiability-assumptions}

\[
Y(a, m), M(a) \perp A | L = l
\]

\[
Y(a, m) \perp M(a) | A, L = l
\]

First holds trivially due to randomization.

Second is strong assumption. \(L\) are \emph{pretreatment} variables.
Further assumptions are needed for time-varying confounding.

\section{Mapping to a statistical
model}\label{mapping-to-a-statistical-model}

\begin{itemize}
\tightlist
\item
  note that we don't make no-interaction assumption.
\end{itemize}

\section{Discussion}\label{discussion}

Worth discussing Suzuki, Yamamoto, and Tsuda (2011) and the sufficient
cause model?

\section*{References}\label{references}
\addcontentsline{toc}{section}{References}

\hypertarget{refs}{}
\hypertarget{ref-fukami2015historical}{}
Fukami, Tadashi. 2015. ``Historical Contingency in Community Assembly:
Integrating Niches, Species Pools, and Priority Effects.'' \emph{Annual
Review of Ecology, Evolution, and Systematics} 46 (1): 1--23.

\hypertarget{ref-halliday2017interactions}{}
Halliday, Fletcher W., James Umbanhowar, and Charles E. Mitchell. 2017.
``Interactions Among Symbionts Operate Across Scales to Influence
Parasite Epidemics.'' \emph{Ecology Letters} 20 (10): 1285--94.

\hypertarget{ref-imai2010general}{}
Imai, Kosuke, Luke Keele, and Dustin Tingley. 2010. ``A General Approach
to Causal Mediation Analysis.'' \emph{Psychological Methods} 15 (4):
309--34.

\hypertarget{ref-richardson2013primer}{}
Richardson, Thomas S., and James M. Robins. 2013. ``Single World
Intervention Graphs: A Primer.''

\hypertarget{ref-shipley2016cause}{}
Shipley, Bill. 2016. \emph{Cause and Correlation in Biology: A User's
Guide to Path Analysis, Structural Equations and Causal Inference with
R}. Cambridge University Press.

\hypertarget{ref-suzuki2011identification}{}
Suzuki, Etsuji, Eiji Yamamoto, and Toshihide Tsuda. 2011.
``Identification of Operating Mediation and Mechanism in the
Sufficient-Component Cause Framework.'' \emph{European Journal of
Epidemiology} 26 (5): 347--57.

\hypertarget{ref-vanderweele2015explanation}{}
VanderWeele, Tyler. 2015. \emph{Explanation in Causal Inference: Methods
for Mediation and Interaction}. Oxford University Press.


\end{document}
