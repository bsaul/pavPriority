\documentclass[]{article}
\usepackage{lmodern}
\usepackage{amssymb,amsmath}
\usepackage{ifxetex,ifluatex}
\usepackage{fixltx2e} % provides \textsubscript
\ifnum 0\ifxetex 1\fi\ifluatex 1\fi=0 % if pdftex
  \usepackage[T1]{fontenc}
  \usepackage[utf8]{inputenc}
\else % if luatex or xelatex
  \ifxetex
    \usepackage{mathspec}
  \else
    \usepackage{fontspec}
  \fi
  \defaultfontfeatures{Ligatures=TeX,Scale=MatchLowercase}
\fi
% use upquote if available, for straight quotes in verbatim environments
\IfFileExists{upquote.sty}{\usepackage{upquote}}{}
% use microtype if available
\IfFileExists{microtype.sty}{%
\usepackage{microtype}
\UseMicrotypeSet[protrusion]{basicmath} % disable protrusion for tt fonts
}{}
\usepackage[margin=1in]{geometry}
\usepackage{hyperref}
\hypersetup{unicode=true,
            pdftitle={Causal Priority Effects},
            pdfborder={0 0 0},
            breaklinks=true}
\urlstyle{same}  % don't use monospace font for urls
\usepackage{graphicx,grffile}
\makeatletter
\def\maxwidth{\ifdim\Gin@nat@width>\linewidth\linewidth\else\Gin@nat@width\fi}
\def\maxheight{\ifdim\Gin@nat@height>\textheight\textheight\else\Gin@nat@height\fi}
\makeatother
% Scale images if necessary, so that they will not overflow the page
% margins by default, and it is still possible to overwrite the defaults
% using explicit options in \includegraphics[width, height, ...]{}
\setkeys{Gin}{width=\maxwidth,height=\maxheight,keepaspectratio}
\IfFileExists{parskip.sty}{%
\usepackage{parskip}
}{% else
\setlength{\parindent}{0pt}
\setlength{\parskip}{6pt plus 2pt minus 1pt}
}
\setlength{\emergencystretch}{3em}  % prevent overfull lines
\providecommand{\tightlist}{%
  \setlength{\itemsep}{0pt}\setlength{\parskip}{0pt}}
\setcounter{secnumdepth}{0}
% Redefines (sub)paragraphs to behave more like sections
\ifx\paragraph\undefined\else
\let\oldparagraph\paragraph
\renewcommand{\paragraph}[1]{\oldparagraph{#1}\mbox{}}
\fi
\ifx\subparagraph\undefined\else
\let\oldsubparagraph\subparagraph
\renewcommand{\subparagraph}[1]{\oldsubparagraph{#1}\mbox{}}
\fi

%%% Use protect on footnotes to avoid problems with footnotes in titles
\let\rmarkdownfootnote\footnote%
\def\footnote{\protect\rmarkdownfootnote}

%%% Change title format to be more compact
\usepackage{titling}

% Create subtitle command for use in maketitle
\newcommand{\subtitle}[1]{
  \posttitle{
    \begin{center}\large#1\end{center}
    }
}

\setlength{\droptitle}{-2em}

  \title{Causal Priority Effects}
    \pretitle{\vspace{\droptitle}\centering\huge}
  \posttitle{\par}
    \author{}
    \preauthor{}\postauthor{}
    \date{}
    \predate{}\postdate{}
  
\usepackage{mathtools}
\usepackage{tikz}
\usepackage{graphicx}
\usepackage{float}
\usepackage{xcolor}
\usepackage{setspace}
\newcommand*\circled[1]{\tikz[baseline=(char.base)]{ \node[shape=circle,draw,inner sep=2pt] (char) {#1};}}

\begin{document}
\maketitle
\begin{abstract}
TBD
\end{abstract}

\doublespacing

\hypertarget{introduction}{%
\section{Introduction}\label{introduction}}

The early bird gets the worm, but the second mouse gets the cheese.
Despite being poorly understood and studied, priority effects -- the
order and timing of species arrival -- have ecological consequences
(Fukami 2015). Priority effects invite counterfactual reasoning: if
species A arrives first would outcomes for species B be different
compared to a world where species B arrived first? Such counterfactual
questions are well-suited for the potential framework of causal
inference (VanderWeele 2015).

To study within-host priority effects, Halliday, Umbanhowar, and
Mitchell (2017) conducted an experiment wherein three cohorts of
sentinel, pathogen-naive tall fescue (\emph{Festuca arundinacea}) plants
where placed in a field at three different timepoints during the year.
The timepoints were chosen to correspond to periods of high prevelance
during the annual epidemic cycles for each of three tall fescue fungal
pathogens: \emph{Colletotrichum} (early summer), \emph{Rhizoctonia} (mid
summer), and \emph{Puccinia} (late summer) (Halliday, Umbanhowar, and
Mitchell 2017).

In this paper, priority effects are defined as contrasts of potential
outcomes, hence under assumptions that we detail these effects have a
causal interpretation and may be estimated from observed data.

\hypertarget{priority-effects}{%
\subsection{Priority Effects}\label{priority-effects}}

Describe priority effects and importance to ecology (based on Fukami
(2015))

\begin{itemize}
\tightlist
\item
  would be nice to map one of the figures in Fukami (2015) to the causal
  graph
\end{itemize}

\hypertarget{causal-inference-using-potential-outcomes}{%
\subsection{Causal inference using potential
outcomes}\label{causal-inference-using-potential-outcomes}}

In this paper, causal research questions about priority effects are
explicity defined in terms of potential outcomes. Using a potential
outcome approach, causal parameters can be defined and estimators
derived without reference to specific statistical models (Imai, Keele,
and Tingley 2010).

\hypertarget{discrete-time-causal-models-for-within-host-priority-effects}{%
\section{Discrete-Time Causal Models for Within-Host Priority
Effects}\label{discrete-time-causal-models-for-within-host-priority-effects}}

Define \(\mathsf{I}_{jkt} = (C_{jkt}, R_{jkt}, P_{jkt})^\intercal\) as
the indicators of observed infection status by pathogens
\emph{Colletotrichum}, \emph{Rhizoctonia}, and \emph{Puccinia} at week
\(t = 1, \dots,\) after the start of the experiment on leaf
\(k = 1, \dots, n_{jt}\) in plant \(j = 1, \dots, m\). Let \(D_{jkt}\)
be an indicator that a leaf is alive (1) or not (0) at \(t\);
\(H_{j} \in \{1, 2, 3\}\) indicate plant \(j\)'s randomly assigned
cohort; \(A_{jkt}\) be a leaf's age (in weeks) on week \(t\); and
\(L_{jkt}\) be additional covariates such as abiotic factors measured in
the time up to the timepoint \(t\),

Denote the random matrix of infection exposures for all \(n_{jt}\)
leaves in plant \(j\) at time \(t\) as
\(\mathsf{I}_{kt} = (\mathsf{I}_{j1t} , \dots, \mathsf{I}_{kn_jt})\),
and let \(\mathsf{I}_{j \setminus k, t}\) be the submatrix of
\(\mathsf{I}_{jt}\) of exposure for all leaves other than the \(j\)th
one. Lower-case letters denote possible realizations of
\(\mathsf{I}_{jkt}\), \(\mathsf{I}_{j \setminus k, t}\), and
\(\mathsf{I}_{k}\): \(\mathsf{i}_{jkt}\),
\(\mathsf{i}_{j \setminus k, t}\) and \(\mathsf{i}_{kt}\), respectively.
Define
\(\omega(\mathsf{i}_{t}; \rho) = \Pr_{\rho}(\mathsf{I}_{jt} = \mathsf{i}_{t})\)
as the marginal probability under within-plant pressure
\(\rho = (\rho^c_{t-1}, \rho^r_{t-1}, \rho^p_{t-1})\) that a host plant
experiences \(\mathsf{i}_t\) at time \(t\).

A goal of the model is to make comparisons such as:

\begin{itemize}
\tightlist
\item
  How does the risk of \emph{Colletotrichum} infection change had the
  leaf not been infected by \emph{Rhizoctonia} at the previous timepoint
  \emph{versus} not having been infected by \emph{Rhizoctonia}, holding
  within-plant \emph{Rhizoctonia} intensity and within-plant
  \emph{Colletotrichum} epidemic intensity fixed?
\item
  How does the risk of \emph{Rhizoctonia} infection change had the
  within-plant \emph{Rhizoctonia} intensity been \(\rho_r\) versus
  \(\rho_r'\) and within-plant \emph{Colletotrichum} intensity fixed and
  not having been infected by \emph{Colletotrichum}?
\end{itemize}

\hypertarget{causal-effects-estimands}{%
\subsection{Causal effects (estimands)}\label{causal-effects-estimands}}

For each \(t\), the average potential outcomes are defined for each
pathogen under the case that the pathogen infection statuses had been
\(\mathsf{i}_{t - 1} = (c_{t -1}, r_{t-1}, p_{t-1})\) and the
within-host pressure had been \(\rho\) in the previous week.
Additionally, this estimand is conditional upon not having add been
previously infected by the same pathogen and not having died. Cohort
assignment and leaf age are incorporated as effect modifiers, as cohort
timing and leaf age are relevant to research questions.

NOTES:

\begin{itemize}
\tightlist
\item
  Leaf death is a competing risk to infection. However, some of the
  pathogens may induce death. So, incorporate this competing risk, or
  just define (as here) population conditional upon leaf not dying
\item
  as we may expect the effects to be different across cohorts, we define
  the effects to be different. But we do not define the potential
  outcomes as functions of the cohort since we do not think that the
  underlying causal processes are dependent on the timing of the
  cohort's placement in the field. Allows for effect modification by
  leaf age.
\end{itemize}

\begin{align*}
\mu_{aht}(\mathsf{i}_{t-1}, \rho) = 
\begin{pmatrix}
  \mu^c_{aht}(\mathsf{i}_{t-1}, \rho) \\
  \mu^r_{aht}(\mathsf{i}_{t-1}, \rho) \\
  \mu^p_{aht}(\mathsf{i}_{t-1}, \rho) 
\end{pmatrix} &= 
\mathrm{E} 
\left\{ 
\begin{array}{l}
  \bar{C}_{t}(\mathsf{i}_{t-1}, \rho) | A = a, H = h, C_{jk,t-1} = 0, D_{jkt} = 0 \\
  \bar{R}_{t}(\mathsf{i}_{t-1}, \rho) | A = a, H = h, R_{jk,t-1} = 0, D_{jkt} = 0 \\ 
  \bar{P}_{t}(\mathsf{i}_{t-1}, \rho) | A = a, H = h, P_{jk,t-1} = 0, D_{jkt} = 0
\end{array}
\right\} \\
&= 
\mathrm{E} \left\{
\begin{array}{l}
  \sum_{\mathsf{i}_{j \setminus k}} C_{t}(\mathsf{i}_{t-1}; \rho) \omega(\mathsf{i}_{t-1}; \rho) | A = a, H = h, C_{t-1} = 0, D_t = 0 \\
  \sum_{\mathsf{i}_{j \setminus k}} R_{t}(\mathsf{i}_{t-1}; \rho) \omega(\mathsf{i}_{t-1}; \rho) | A = a, H = h, R_{t-1} = 0, D_t = 0 \\ 
  \sum_{\mathsf{i}_{j \setminus k}} P_{t}(\mathsf{i}_{t-1}; \rho) \omega(\mathsf{i}_{t-1}; \rho) | A = a, H = h, P_{t-1} = 0, D_t = 0
\end{array}
\right\}
,
\end{align*}

where
\(\sum_{\mathsf{i}_{j \setminus k}} \equiv \sum_{\mathbf{r}_{j \setminus k}} \sum_{\mathbf{c}_{j \setminus k}} \sum_{\mathbf{p}_{j \setminus k}}\)
and \(\mathrm{E}(\cdot)\) is the expected value in the super-population
of plants.

Table \ref{tab:estimands} provides a typology of estimands under
consideration for this study. For example,
\(AL[(1, 0, 0)^{\intercal}, (0, 0, 0)^{\intercal}, \rho]\) would compare
the average potential outcomes under the setting where a leaf is
infected by \emph{Colletotrichum} and nothing else in the previous week
versus the setting where leaves where not infected by any pathogen, for
a fixed \(\rho\). In essense, this is a comparison of the probability of
infection by each of three pathogens in week \(t\) under two different
settings of the previous week: a leaf is infected by
\emph{Colletotrichum} only versus no infections with the prevelance of
infections within the plant held at \(\rho\).

Under this setup, a variety of interesting causal questions can be
asked. Whether and how these quantities can be estimated from data will
be considered in Section XXX.

\begin{table}[H]
\caption{A typology of estimands for this study}
\label{tab:estimands}
\begin{tabular}{llll}
Label & Name of effect & Causal contrast of $\mu_{aht}(\cdot)$ & Description \\
\hline
$AL(\mathsf{i}_{t-1}, \mathsf{i}'_{t-1}, \rho)$ & Average Leaf-level &  $\cdot = (\mathsf{i}_{t-1}, \rho)$ vs $(\mathsf{i}'_{t-1}, \rho)$ & \\
$AS(\mathsf{i}_{t-1}, \rho, \rho')$ & Average Within-plant Spillover & $(\mathsf{i}_{t-1}, \rho)$ vs $(\mathsf{i}_{t-1}, \rho')$ & \\
$AT(\mathsf{i}_{t-1}, \mathsf{i}'_{t-1}, \rho, \rho')$  & Average Total & $(\mathsf{i}_{t-1}, \rho)$ vs $(\mathsf{i}'_{t-1}, \rho')$ & \\
\end{tabular}
\end{table}

\begin{figure}
\centering
\includegraphics{causal_priority_effects_V012_files/figure-latex/tikz-def-1.pdf}
\caption{Defintion for \(I_t\) graph}
\end{figure}

\begin{figure}
\centering
\includegraphics{causal_priority_effects_V012_files/figure-latex/tikz-swit-1.pdf}
\caption{Intervention template for a single plant evolving over time.
Time two timepoints are shown. The plant has two leaves in the first
week, and third leaf emerges in the second week.}
\end{figure}

\hypertarget{assumptions-of-this-causal-model}{%
\subsection{Assumptions of this causal
model}\label{assumptions-of-this-causal-model}}

\begin{itemize}
\tightlist
\item
  no carryover effects (Markov assumption)
\end{itemize}

\hypertarget{experiment}{%
\subsection{Experiment}\label{experiment}}

Describe the experiment that was run:

\begin{itemize}
\tightlist
\item
  experimental treatments
\item
  what was measured
\end{itemize}

Describe study predictions or hypotheses.

\hypertarget{estimation}{%
\subsection{Estimation}\label{estimation}}

Exposure model:

\begin{itemize}
\tightlist
\item
  good case for mixed-effects multivariate Bernoulli model (Dai, Ding,
  and Wahba 2013)? i.e., model the probability of an individual leaf's
  infection statuses as:
\end{itemize}

\[
h\{\Pr(C_{jkt} = c, R_{jkt}  = r, P_{jkt}  = p | L_{jkt}, b_{jt})|\}
\]

where \(h\) is (e.g.) the inverse logistic function. Then maybe estimate
the plant's infection status as:

\[
\Pr(\mathsf{I}_{jt} = \mathsf{i}_{jt} | L_{jt}) = \int \prod_{k = 1}^{n_j}h\{\Pr(\mathsf{I}_{jkt} = \mathsf{i}_{jkt}  | L_{jkt}, b_{jt})\}^{I(\mathsf{I}_{jkt} = \mathsf{i}_{jkt} )} f(b_{jt}) \mathsf{d}b_{jt}
\]

\begin{itemize}
\tightlist
\item
  Description of statistical models

  \begin{itemize}
  \tightlist
  \item
    Barkley et al. (2017)
  \item
    account for ordering of leaves within host
  \end{itemize}
\item
  M-estimation for variance estimates
\end{itemize}

\hypertarget{results}{%
\section{Results}\label{results}}

\begin{itemize}
\tightlist
\item
  Sensitively + stability analyses
\end{itemize}

\hypertarget{discussion}{%
\section{Discussion}\label{discussion}}

\begin{itemize}
\tightlist
\item
  Discussion on mechanisms:

  \begin{itemize}
  \tightlist
  \item
    Data from this experiment cannot distinguish between niche
    preemption or niche modification. Needs additional assumptions and
    knowledge, and this is even more complicated if a species can change
    from (e.g.) biotroph to necrotroph.
  \item
    mediation - how an effect occurs; interaction - for whom an effect
    occurs (VanderWeele 2015)
  \end{itemize}
\item
  suggestions for better designs and how identifiability results might
  suggest better design
\item
  links to other causal frameworks
\item
  link back to Halliday, Umbanhowar, and Mitchell (2017)

  \begin{itemize}
  \tightlist
  \item
    is death of leaf a censoring event or competing risk? There's a
    complication that leaves only live a certain period of time - during
    which they are either infected or not - but it is not as if a leaf
    \emph{could} survive for longer during which time we could observe
    the outcome. Then there's the fact that leaves in the last cohort
    can't be observed for the same length of time as other cohorts due
    to end of study and senesence.
  \end{itemize}
\item
  Structural equation models (Shipley 2016) are popular in biology and
  ecology; however, parameters from these models may not carry a causal
  interpretation without assumptions often implicitly made (Imai, Keele,
  and Tingley 2010). Similarly, it is not uncommon for authors to
  discuss the results of associational statistical models in terms of
  causal language.
\end{itemize}

\hypertarget{references}{%
\section*{References}\label{references}}
\addcontentsline{toc}{section}{References}

\hypertarget{refs}{}
\leavevmode\hypertarget{ref-barkley2017causal}{}%
Barkley, Brian G, Michael G Hudgens, John D Clemens, Mohammad Ali, and
Michael E Emch. 2017. ``Causal Inference from Observational Studies with
Clustered Interference.'' \emph{arXiv Preprint arXiv:1711.04834}.

\leavevmode\hypertarget{ref-dai2013multivariate}{}%
Dai, Bin, Shilin Ding, and Grace Wahba. 2013. ``Multivariate Bernoulli
Distribution.'' \emph{Bernoulli} 19 (4): 1465--83.

\leavevmode\hypertarget{ref-fukami2015historical}{}%
Fukami, Tadashi. 2015. ``Historical Contingency in Community Assembly:
Integrating Niches, Species Pools, and Priority Effects.'' \emph{Annual
Review of Ecology, Evolution, and Systematics} 46 (1): 1--23.

\leavevmode\hypertarget{ref-halliday2017interactions}{}%
Halliday, Fletcher W., James Umbanhowar, and Charles E. Mitchell. 2017.
``Interactions Among Symbionts Operate Across Scales to Influence
Parasite Epidemics.'' \emph{Ecology Letters} 20 (10): 1285--94.

\leavevmode\hypertarget{ref-imai2010general}{}%
Imai, Kosuke, Luke Keele, and Dustin Tingley. 2010. ``A General Approach
to Causal Mediation Analysis.'' \emph{Psychological Methods} 15 (4):
309--34.

\leavevmode\hypertarget{ref-shipley2016cause}{}%
Shipley, Bill. 2016. \emph{Cause and Correlation in Biology: A User's
Guide to Path Analysis, Structural Equations and Causal Inference with
R}. Cambridge University Press.

\leavevmode\hypertarget{ref-vanderweele2015explanation}{}%
VanderWeele, Tyler. 2015. \emph{Explanation in Causal Inference: Methods
for Mediation and Interaction}. Oxford University Press.


\end{document}
