\documentclass[]{article}
\usepackage{lmodern}
\usepackage{amssymb,amsmath}
\usepackage{ifxetex,ifluatex}
\usepackage{fixltx2e} % provides \textsubscript
\ifnum 0\ifxetex 1\fi\ifluatex 1\fi=0 % if pdftex
  \usepackage[T1]{fontenc}
  \usepackage[utf8]{inputenc}
\else % if luatex or xelatex
  \ifxetex
    \usepackage{mathspec}
  \else
    \usepackage{fontspec}
  \fi
  \defaultfontfeatures{Ligatures=TeX,Scale=MatchLowercase}
\fi
% use upquote if available, for straight quotes in verbatim environments
\IfFileExists{upquote.sty}{\usepackage{upquote}}{}
% use microtype if available
\IfFileExists{microtype.sty}{%
\usepackage{microtype}
\UseMicrotypeSet[protrusion]{basicmath} % disable protrusion for tt fonts
}{}
\usepackage[margin=1in]{geometry}
\usepackage{hyperref}
\hypersetup{unicode=true,
            pdftitle={Causal Priority Effects},
            pdfborder={0 0 0},
            breaklinks=true}
\urlstyle{same}  % don't use monospace font for urls
\usepackage{graphicx,grffile}
\makeatletter
\def\maxwidth{\ifdim\Gin@nat@width>\linewidth\linewidth\else\Gin@nat@width\fi}
\def\maxheight{\ifdim\Gin@nat@height>\textheight\textheight\else\Gin@nat@height\fi}
\makeatother
% Scale images if necessary, so that they will not overflow the page
% margins by default, and it is still possible to overwrite the defaults
% using explicit options in \includegraphics[width, height, ...]{}
\setkeys{Gin}{width=\maxwidth,height=\maxheight,keepaspectratio}
\IfFileExists{parskip.sty}{%
\usepackage{parskip}
}{% else
\setlength{\parindent}{0pt}
\setlength{\parskip}{6pt plus 2pt minus 1pt}
}
\setlength{\emergencystretch}{3em}  % prevent overfull lines
\providecommand{\tightlist}{%
  \setlength{\itemsep}{0pt}\setlength{\parskip}{0pt}}
\setcounter{secnumdepth}{0}
% Redefines (sub)paragraphs to behave more like sections
\ifx\paragraph\undefined\else
\let\oldparagraph\paragraph
\renewcommand{\paragraph}[1]{\oldparagraph{#1}\mbox{}}
\fi
\ifx\subparagraph\undefined\else
\let\oldsubparagraph\subparagraph
\renewcommand{\subparagraph}[1]{\oldsubparagraph{#1}\mbox{}}
\fi

%%% Use protect on footnotes to avoid problems with footnotes in titles
\let\rmarkdownfootnote\footnote%
\def\footnote{\protect\rmarkdownfootnote}

%%% Change title format to be more compact
\usepackage{titling}

% Create subtitle command for use in maketitle
\newcommand{\subtitle}[1]{
  \posttitle{
    \begin{center}\large#1\end{center}
    }
}

\setlength{\droptitle}{-2em}

  \title{Causal Priority Effects}
    \pretitle{\vspace{\droptitle}\centering\huge}
  \posttitle{\par}
    \author{}
    \preauthor{}\postauthor{}
    \date{}
    \predate{}\postdate{}
  
\usepackage{mathtools}
\usepackage{tikz}
\usepackage{graphicx}
\usepackage{xcolor}
\newcommand*\circled[1]{\tikz[baseline=(char.base)]{ \node[shape=circle,draw,inner sep=2pt] (char) {#1};}}

\begin{document}
\maketitle
\begin{abstract}
TBD
\end{abstract}

\hypertarget{introduction}{%
\section{Introduction}\label{introduction}}

The early bird gets the worm, but the second mouse gets the cheese.
Despite being poorly understood and studied, priority effects -- the
order and timing of species arrival -- have ecological consequences
(Fukami 2015). Priority effects invite counterfactual reasoning: if
species A arrives first would outcomes for species B be different
compared to a world where species B arrived first? Such counterfactual
questions are well suited for the potential framework of causal
inference (VanderWeele 2015).

To study within-host priority effects, Halliday, Umbanhowar, and
Mitchell (2017) conducted an experiment wherein three cohorts of
sentinel, pathogen-na:\{i\}ve tall fescue (\emph{Festuca arundinacea})
plants where placed in a field at three different timepoints during the
year. The timepoints were chosen to correspond to periods of high
prevelance during the annual epidemic cycles for each of three tall
fescue fungal pathogens: \emph{Colletotrichum} (early summer),
\emph{Rhizoctonia} (mid summer), and \emph{Puccinia} (late summer)
(Halliday, Umbanhowar, and Mitchell 2017).

A goal of this paper is to introduce experimental ecologist to the
causal inference methods of XXX, YYY, ZZZ. To do so, we begin by
expressing priority effects as functions of potential outcomes, thus
giving contrasts of potential outcomes a causal interpretation. We
discuss the assumptions necessary for identifying and estimating such
effects from observed data. Causal inference methods more familiar to
many ecologists such as structural equation models make certain implicit
assumptions, which, if violated, may lead to incorrect conclusions
(Imai, Keele, and Tingley 2010). After establishing the targets of
inference (estimands) and assumptions necessary for causal inference,
statistical models are parameterized to target the estimands and
estimatation and inference carried out.

\hypertarget{priority-effects}{%
\subsection{Priority Effects}\label{priority-effects}}

Describe priority effects and importance to ecology (based on Fukami
(2015))

\begin{itemize}
\tightlist
\item
  Is hypothesis in this experiment niche preemption or niche
  modification? I think modification, though this complicated if a
  species can change from (e.g.) biotroph to necrotroph.
\item
  link priority effects to causal mediation:

  \begin{itemize}
  \tightlist
  \item
    mediation - how an effect occurs; interaction - for whom an effect
    occurs (VanderWeele 2015)
  \item
    would be nice to map one of the figures in Fukami (2015) to the
    causal mediation SWIT
  \end{itemize}
\end{itemize}

Wrinkle with this study/data

\begin{itemize}
\tightlist
\item
  randomized exposure (time of field placement) is essentially an
  instrumental variable
\item
  interval censored data
\item
  is death of leaf a censoring event or competing risk? There's a
  complication that leaves only live a certain period of time - during
  which they are either infected or not - but it is not as if a leaf
  \emph{could} survive for longer during which time we could observe the
  outcome. Then there's the fact that leaves in the last cohort can't be
  observed for the same length of time as other cohorts due to end of
  study and senesence.
\end{itemize}

\hypertarget{causal-inference-using-potential-outcomes}{%
\subsection{Causal inference using potential
outcomes}\label{causal-inference-using-potential-outcomes}}

Structural equation models (Shipley 2016) are popular in biology and
ecology; however, parameters from these models may not carry a causal
interpretation without assumptions often implicitly made (Imai, Keele,
and Tingley 2010).

\begin{itemize}
\tightlist
\item
  the approach we describe starts from causal questions and works
  towards statistical ``answers'' (i.e.~parameter estimates, confidence
  intervals, hypothesis tests, etc)
\item
  the typical approach, however, is to take statistical ``answers'' and
  assume they are causal
\item
  a more principled approach is warranted and available
\item
  Imai, Keele, and Tingley (2010) argue the structural equation modeling
  approach to mediation ``is problematic for 3 reasons: the lack of a
  general definition of causal mediation effects independent of a
  particular statistical model, the inability to specify the key
  identification assumption, and the difficulty of extending the
  framework to nonlinear models.''
\end{itemize}

Problems with linear structural equation models (Imai, Keele, and
Tingley 2010):

\begin{itemize}
\tightlist
\item
  within LSEM, causal mediation effects are defined relative to
  particular statistical models
\item
  not generalization to nonlinear models {[}is this true for the SEM
  framework?{]}
\end{itemize}

An advantage of the potential outcome approach is that the causal
effects can be defined, identified, and estimators derived without
reference to specific statistical models (Imai, Keele, and Tingley
2010).

\hypertarget{a-causal-model-for-priority-effects}{%
\section{A Causal Model for Priority
Effects}\label{a-causal-model-for-priority-effects}}

\begin{itemize}
\tightlist
\item
  First describe the target causal estimand; introduce notation
\item
  Then describe an idealized randomized experiment from which the
  estimand could be straightfowardly estimated
\item
  Then describe issues with this experiment (why it is difficult to
  carry out)
\item
  Then describe the experiment that was conducted in Halliday,
  Umbanhowar, and Mitchell (2017)
\end{itemize}

\hypertarget{experiment}{%
\subsection{Experiment}\label{experiment}}

Describe the experiment that was run:

\begin{itemize}
\tightlist
\item
  experimental treatments
\item
  what was measured
\end{itemize}

Describe study predictions or hypotheses.

\hypertarget{notation}{%
\subsection{Notation}\label{notation}}

Let \(Y^p_{ij}\) be an indicator of observed visible symptoms of
pathogen \(p\) at 4 weeks after leaf emergence in sentinel plant (host)
\(i = 1, \dots, m\) and leaf \(j = 1, \dots, n_i\). Let \(A^{p'}_{ij}\)
be an indicator of infection (confirmed by visible symptoms) by pathogen
\(p'\) during the time between leaf emergence and up to three weeks
after emergence but before any indication of pathogen \(p\) infection
during that time. Let \(S^{p'}_{ij}\) be the the proportion of leaves in
plant \(i\), not including leaf \(j\) infected by pathogen \(p'\) at the
time of leaf \(j\) emergence. Let \(W^{p}_{ij}\) and \(W^{p'}_{ij}\)be
the average number of leaves infected by pathogen \(p\) and \(p'\),
respectively in other experimental plants other than plant \(i\) at the
time of leaf \(j\) emergence. Let \(L_{ij}\) be other a vector of other
baseline covariates measured at the time of leaf \(j\) emergence, such
as calendar time.

Define the average potential infection of leaf \(j\) by pathogen \(p\)
had it been exposed to pathogen \(p'\) and within host prevelance at
emergence as
\(\bar{Y}_{ij} = \sum_{\mathcal{A}} Y^p_{ij}(a^{p'}, s^{p'}) \omega(\alpha; \cdot)\)
{[}TODO: define \(\mathcal{A}\) and \(\omega\) and \(\alpha\){]}. Define
the average \emph{within plant} potential outcome as
\(\bar{Y}^p_i(a^{p'}, s^{p'}) = \sum_{j}^{n_i} Y^p_{ij}(a^{p'}, s^{p'})\)
and average \emph{population} potential outcome as
\(\bar{Y}^p(a^{p'}, s^{p'}) = \sum_{i}^{m} \bar{Y}^p_{i}(a^{p'}, s^{p'})\).

\begin{itemize}
\tightlist
\item
  Assume no interference between plants (clustered interference).
  Explain why this is/isn't reasonable.
\item
  I've dropped the tiller level here. Discuss.
\item
  TODO: add cohort; add as effect modifier? baseline covariate?
\end{itemize}

\hypertarget{causal-effects-estimands}{%
\subsubsection{Causal effects
(estimands)}\label{causal-effects-estimands}}

\begin{itemize}
\tightlist
\item
  Average within leaf priority effect:
  \(LPE^p(\alpha) = \bar{Y}^p(1; \alpha) - \bar{Y}^p(0; \alpha)\). This
  the average effect of prior infection by pathogen \(p'\) on subsequent
  infection by pathogen \(p\) within a leaf, under assignment scheme
  \(\alpha\). A positive value suggests prior \(p'\) infection
  facilitates subesequent \(p\) infection; a negative value suggests
  \(p'\) reduces \(p\) infection.
\item
  Average uninfected within host priority effect:
  \(UHPE^p(0, \alpha, \alpha') = \bar{Y}^p(0; \alpha) - \bar{Y}^p(0; \alpha')\).
  This is the average effect of modifying within host assignment scheme
  from \(\alpha\) to \(\alpha'\) in leaves not infected by \(p'\).
\item
  Average infected within host priority effect:
  \(IHPE^p(1, \alpha, \alpha') = \bar{Y}^p(1; \alpha) - \bar{Y}^p(1; \alpha')\).
  This is the average effect of modifying within host assignment scheme
  from \(\alpha\) to \(\alpha'\) in leaves infected by \(p'\).
\end{itemize}

\hypertarget{estimation}{%
\subsection{Estimation}\label{estimation}}

\begin{itemize}
\tightlist
\item
  Description of statistical models

  \begin{itemize}
  \tightlist
  \item
    Barkley et al. (2017)
  \item
    account for ordering of leaves within host
  \end{itemize}
\item
  M-estimation for variance estimates
\end{itemize}

\hypertarget{results}{%
\section{Results}\label{results}}

\begin{itemize}
\tightlist
\item
  Sensitively + stability analyses
\end{itemize}

\hypertarget{discussion}{%
\section{Discussion}\label{discussion}}

\begin{itemize}
\tightlist
\item
  discuss the exposure. Is this well-defined? In this current setup does
  it partly depends on the outcome?
\item
  a call for more causal in causal
\item
  suggestions for better designs and how identifiability results might
  suggest better design
\item
  links to other causal frameworks
\end{itemize}

\hypertarget{references}{%
\section*{References}\label{references}}
\addcontentsline{toc}{section}{References}

\hypertarget{refs}{}
\leavevmode\hypertarget{ref-barkley2017causal}{}%
Barkley, Brian G, Michael G Hudgens, John D Clemens, Mohammad Ali, and
Michael E Emch. 2017. ``Causal Inference from Observational Studies with
Clustered Interference.'' \emph{arXiv Preprint arXiv:1711.04834}.

\leavevmode\hypertarget{ref-fukami2015historical}{}%
Fukami, Tadashi. 2015. ``Historical Contingency in Community Assembly:
Integrating Niches, Species Pools, and Priority Effects.'' \emph{Annual
Review of Ecology, Evolution, and Systematics} 46 (1): 1--23.

\leavevmode\hypertarget{ref-halliday2017interactions}{}%
Halliday, Fletcher W., James Umbanhowar, and Charles E. Mitchell. 2017.
``Interactions Among Symbionts Operate Across Scales to Influence
Parasite Epidemics.'' \emph{Ecology Letters} 20 (10): 1285--94.

\leavevmode\hypertarget{ref-imai2010general}{}%
Imai, Kosuke, Luke Keele, and Dustin Tingley. 2010. ``A General Approach
to Causal Mediation Analysis.'' \emph{Psychological Methods} 15 (4):
309--34.

\leavevmode\hypertarget{ref-shipley2016cause}{}%
Shipley, Bill. 2016. \emph{Cause and Correlation in Biology: A User's
Guide to Path Analysis, Structural Equations and Causal Inference with
R}. Cambridge University Press.

\leavevmode\hypertarget{ref-vanderweele2015explanation}{}%
VanderWeele, Tyler. 2015. \emph{Explanation in Causal Inference: Methods
for Mediation and Interaction}. Oxford University Press.


\end{document}
